\setuppapersize[A4]
\setupbodyfont[dejavu]

\mainlanguage[russian]
\language[ru]

% нумерация
\setupfootertexts[]
\setuppagenumbering[state=start, location=bottom]

% Главы
\setuphead[section][
  style=\bfb,
  indentnext=yes,
]

\setuphead[subsection][
  style=\bfa,
  indentnext=yes,
]
\setupindenting[yes,medium]
\setuplabeltext[ru][appendix={Приложение }]

\starttext

\startalignment[middle, wide]
{\bfd Лабораторная работа №2} \blank[big]
{\tfb "Работа с данными. Простые запросы на выборку"} \blank[4*big]
{\tfb Выполнил студент группы Б20-505} \blank[big]
{\bfb Сорочан Илья}
{\vfill}
{\tfa Московский Инженерно-Физический Институт} \blank[big] % здесь можно указать факультет или кафедру
{\tfa Москва 2023} \blank[big] % здесь можно указать факультет или кафедру
\stopalignment

\page

\startsection[title={Введение}]

В данной лабораторной работе будут разработаны различные запросы. Запросы будут разобраны в самой лабораторной. Скрипт по созданию тестовой ДБ можно найти в приложении или в репозитории.

\stopsection

\page

\startsection[title={Выполненные SQL запросы}]
\startsubsection[title={Полная выборка}]

В качестве примера выведем список доступных категорий товаров:
\starttyping
sqlite> select name from category;
Ручки
Тетради
Карандаши
\stoptyping

\stopsubsection

\startsubsection[title={Условная выборка}]

Должности, на которых зарплата больше 2000:
\starttyping
sqlite> select name from job where job.salary > 2000;
Менеджер
Кассир
\stoptyping

\stopsubsection

\startsubsection[title={Упорядоченная выборка}]

Выведем должности, но теперь в порядке увеличения зарплат:
\starttyping
sqlite> select name from job order by job.salary;
Уборщик
Кассир
Менеджер
\stoptyping

Наоборот:
\starttyping
sqlite> select name from job order by job.salary desc;
Менеджер
Кассир
Уборщик
\stoptyping

\stopsubsection

\startsubsection[title={Простые функции и операторы}]

Посмотрим, на каких должностях больше всего подневная зарплата:
\starttyping
select name, job.salary / job.days_per_week from job order by job.salary desc;
\stoptyping

\stopsubsection

\startsubsection[title={Агрегатные функции}]

Найдем самые дорогие товары из каждой страны:
\starttyping
sqlite> select country,max(price) from product group by country;
1|999
2|1299
3|1499
\stoptyping

\stopsubsection

\startsubsection[title={Выбор уникальных записей}]

Представим ситуацию, что мы хотим посмотреть какие работники отвечали за поставки за все время:
\starttyping
sqlite> select distinct employee from supply;
1
2
3
\stoptyping

Для того, что бы вместо цифр были имена можно склеить две таблицы:
\starttyping
sqlite> select distinct name from employee join supply on employee.id = supply.employee;
Джон Стоун
Джек Воробей
Партос
\stoptyping

\stopsubsection

\startsubsection[title={Псевдо-выборка}]

\starttyping
sqlite> select 2*2;
\stoptyping

\stopsubsection

\startsubsection[title={VACUUM}]

\starttyping
sqlite> VACUUM;
\stoptyping

\stopsubsection

\stopsection

\page

\startsection[title={Заключение}]

В ходе данной лабораторной работы было выполнено 8 SQL запросов. Все запросы проиллюстрированы и разобраны в самой работе. Код для создания тестовой базы данных можно найти в приложении или на репозитории.

\stopsection

\startappendices

\startchapter[title={}]

Код для создания тестовой базы данных:
\typefile{setup.sql}

\stopchapter

\stopappendices

\stoptext
