\setuppapersize[A4]
\setupbodyfont[dejavu]

\mainlanguage[russian]
\language[ru]

% нумерация
\setupfootertexts[]
\setuppagenumbering[state=start, location=bottom]

% Главы
\setuphead[section][
  style=\bfb,
  indentnext=yes,
]

\setuphead[subsection][
  style=\bfa,
  indentnext=yes,
]
\setupindenting[yes,medium]
\setuplabeltext[ru][appendix={Приложение }]

\starttext

\startalignment[middle, wide]
{\bfd Зачетная работа} \blank[big]
{\tfb Выполнил студент группы Б20-505} \blank[big]
{\bfb Сорочан Илья}
{\vfill}
{\tfa Московский Инженерно-Физический Институт} \blank[big] % здесь можно указать факультет или кафедру
{\tfa Москва 2023} \blank[big] % здесь можно указать факультет или кафедру
\stopalignment

\page

\startsection[title={Задание}]

Задача данной работы: подробно описать процесс создания сложного SQL запроса к ранее разработанной в лабороторных базе данных.

Для каждого сотрудника вывести таблицу с:
\startitemize[n][start=1]
  \item именем;
  \item названием должности;
  \item общей суммой продаж (в которых он кассир);
  \item число различных типов товаров, среди проданных им;
  \item наименование самой распространенной категории товаров среди проданных им.
\stopitemize

В следующей главе рассмотрим каждый пункт по порядку

\stopsection

\page

\startsection[title={Запросы}]
\startsubsection[title={Имя и название должности}]

Первые два пункта получить относительно просто:
\starttyping
SELECT
  employee.name, job.name
FROM
  employee, job
WHERE
  employee.job = job.id;
\stoptyping

Данный запрос выдаст:
\starttyping
Джон Стоун|Менеджер
Джек Воробей|Кассир
Партос|Уборщик
\stoptyping

\stopsubsection

\startsubsection[title={Общая сумма продаж}]

Здесь уже сложнее, так как требуется обработать данные из трех таблиц. В первую очередь нас интересует таблица sale, в которой хранятся все продажи. Из них мы создаем группы (окна) по продавцу. При это для суммирования цен нам понадобится таблица product, которая хранит айди покупки, к которой принадлежит. И наконец после получения результата его нужно будет объеденить с запросом из предыдущего подраздела.

Запрос:
\starttyping
SELECT
  employee.name, SUM(product.price)
FROM
  sale
JOIN
  employee ON sale.employee = employee.id
JOIN
  product ON sale.id = product.sale
GROUP BY
  employee.id;
\stoptyping

Результат:
\starttyping
Джон Стоун|899
Джек Воробей|1299
Партос|129
\stoptyping

Здесь мы сначала связываем в одну таблицу информацию о продажах, рабочих и товарах, а затем используем агрегатную функцию SUM с разделением по айди рабочих (основной ключ).

Теперь объединим данный запрос с запросом из предыдущей секции:
\starttyping
SELECT
  employee.name, job.name, SUM(product.price), COUNT(DISTINCT product.type)
FROM
  sale
JOIN
  employee ON sale.employee = employee.id
JOIN
  product ON sale.id = product.sale
JOIN
  job ON employee.job = job.id
GROUP BY
  employee.id;
\stoptyping

Выдает следующий результат:
\starttyping
Джон Стоун|Менеджер|899
Джек Воробей|Кассир|1299
Партос|Уборщик|129
\stoptyping

\stopsubsection

\startsubsection[title={Число различных типов товаров}]

Необходимо найти число различных типов товаров, проданных один продавцем. Данный запрос легко выполнить, добавив COUNT(DISTINCT product.type) в предыдущий запрос:
\starttyping
SELECT
  employee.name, job.name, SUM(product.price), COUNT(DISTINCT product.type)
FROM
  sale
JOIN
  employee ON sale.employee = employee.id
JOIN
  product ON sale.id = product.sale
JOIN
  job ON employee.job = job.id
GROUP BY
  employee.id;
\stoptyping

Результат:
\starttyping
Джон Стоун|Менеджер|899|1
Джек Воробей|Кассир|1299|1
Партос|Уборщик|129|1
\stoptyping

\stopsubsection

\startsubsection[title={Самая распространненная категория}]

Последний столбец конечного запроса должен содержать самую распространенную категорию, проданную рабочим.

Начнем с самого простого: получим таблицу и подсчетом всех категорий для всех продаж.
\starttyping
SELECT
  category.id, category.name, COUNT(*)
FROM
  category
JOIN
  product_type ON product_type.category = category.id
JOIN
  product ON product.type = product_type.code
WHERE
  product.sold = true
GROUP BY
  category.id;
\stoptyping

Вывод:
\starttyping
1|Ручки|1
2|Тетради|1
3|Карандаши|1
\stoptyping

Теперь, данную таблицу можно будет использовать в основном запросе, необходимо только добавить столбец с наименованием рабочего-кассира:
\starttyping
SELECT
  category.id, category.name, COUNT(*), sale.employee
FROM
  category
JOIN
  product_type ON product_type.category = category.id
JOIN
  product ON product.type = product_type.code
JOIN
  sale ON product.sale = sale.id
WHERE
  product.sold = true
GROUP BY
  category.id;
\stoptyping

Выдает:
\starttyping
Ручки|1|1
Тетради|1|3
Карандаши|1|2
\stoptyping

Наконец, объеденим данный запрос с предыдущим:
\starttyping
SELECT
  employee.name, job.name, SUM(product.price), COUNT(DISTINCT product.type), category_count_list.category_name
FROM
  sale
JOIN
  employee ON sale.employee = employee.id
JOIN
  product ON sale.id = product.sale
JOIN
  job ON employee.job = job.id
JOIN
  (
    SELECT
      category.name as category_name, COUNT(*) AS category_count, sale.employee
    FROM
      category
    JOIN
      product_type ON product_type.category = category.id
    JOIN
      product ON product.type = product_type.code
    JOIN
      sale ON product.sale = sale.id
    WHERE
      product.sold = true
    GROUP BY
      category.id, sale.employee
    HAVING
      category_count = (
        SELECT
          MAX(subquery.category_count)
        FROM
          (
            SELECT
              COUNT(*) AS category_count, sale.employee
            FROM
              category
            JOIN
              product_type ON product_type.category = category.id
            JOIN
              product ON product.type = product_type.code
            JOIN
              sale ON product.sale = sale.id
            WHERE
              product.sold = true
            GROUP BY
              category.id, sale.employee
          ) AS subquery
        WHERE
          subquery.employee = sale.employee
      )
  ) AS category_count_list ON category_count_list.employee = employee.id
GROUP BY
  employee.id;
\stoptyping

Здесь выполняется два подзапроса: один подсчитывает число категорий, другой отбирает максимальные.

\stopsubsection

\stopsection

\page

\startsection[title={Заключение}]

Задание было выполнено, запрос можно найти в последней секции предыдущей главы. От себя отмечу что коследний столбец был очень сложен, в силу чего я мог сделать неоптимизированный запрос.

\stopsection

\startappendices

\startchapter[title={}]

Код для создания тестовой базы данных:
\typefile{setup.sql}

\stopchapter

\stopappendices

\stoptext
